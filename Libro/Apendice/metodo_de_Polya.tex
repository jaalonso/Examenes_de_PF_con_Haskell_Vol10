\chapter{Método de Pólya para la resolución de problemas}
\label{Polya}

\section{Método de Pólya para la resolución de problemas matemáticos}

\noindent Para resolver un problema se necesita:
\subsubsection*{Paso 1: Entender el problema}
\begin{itemize}
\item ¿Cuál es la incógnita?, ¿Cuáles son los datos?
\item ¿Cuál es la condición? ¿Es la condición suficiente para determinar la
  incógnita? ¿Es insuficiente? ¿Redundante? ¿Contradictoria? 
\end{itemize}

\subsubsection*{Paso 2: Configurar un plan}
\begin{itemize}
\item ¿Te has encontrado con un problema semejante? ¿O has visto el mismo
  problema planteado en forma ligeramente diferente? 
\item ¿Conoces algún problema relacionado con éste? ¿Conoces algún teorema que
  te pueda ser útil? Mira atentamente la incógnita y trata de recordar un
  problema que sea familiar y que tenga la misma incógnita o una incógnita
  similar. 
\item He aquí un problema relacionado al tuyo y que ya has resuelto ya. ¿Puedes
  utilizarlo? ¿Puedes utilizar su resultado? ¿Puedes emplear su método? ¿Te
  hace falta introducir algún elemento auxiliar a fin de poder utilizarlo? 
\item ¿Puedes enunciar al problema de otra forma? ¿Puedes plantearlo en forma
  diferente nuevamente? Recurre a las definiciones.
\item Si no puedes resolver el problema propuesto, trata de resolver primero
  algún problema similar. ¿Puedes imaginarte un problema análogo un tanto más
  accesible? ¿Un problema más general? ¿Un problema más particular? ¿Un problema
  análogo? ¿Puede resolver una parte del problema? Considera sólo una parte de
  la condición; descarta la otra parte; ¿en qué medida la incógnita queda ahora
  determinada? ¿En qué forma puede variar? ¿Puedes deducir algún
  elemento útil de los datos? ¿Puedes pensar en algunos otros datos apropiados
  para determinar la incógnita? ¿Puedes cambiar la incógnita? ¿Puedes cambiar la
  incógnita o los datos, o ambos si es necesario, de tal forma que estén más
  cercanos entre sí? 
\item ¿Has empleado todos los datos? ¿Has empleado toda la condición? ¿Has
  considerado todas las nociones esenciales concernientes al problema? 
\end{itemize}

\subsubsection*{Paso 3: Ejecutar el plan}
\begin{itemize}
\item Al ejercutar tu plan de la solución, comprueba cada uno de los pasos
\item ¿Puedes ver claramente que el paso es correcto? ¿Puedes demostrarlo? 
\end{itemize}

\subsubsection*{Paso 4: Examinar la solución obtenida}
\begin{itemize}
\item ¿Puedes verificar el resultado? ¿Puedes el razonamiento?
\item ¿Puedes obtener el resultado en forma diferente? ¿Puedes verlo de golpe?
  ¿Puedes emplear el resultado o el método en algún otro problema? 
\end{itemize}

\noindent
\textit{G. Polya ``Cómo plantear y resolver problemas'' (Ed. Trillas, 1978)
  p. 19} 

\section{Método de Pólya para resolver problemas de programación}

\noindent Para resolver un problema se necesita:
\subsubsection*{Paso 1: Entender el problema}
\begin{itemize}
\item ¿Cuáles son las \emph{argumentos}? ¿Cuál es el \emph{resultado}? ¿Cuál es
  \emph{nombre} de la función? ¿Cuál es su \emph{tipo}?
\item ¿Cuál es la \emph{especificación} del problema? ¿Puede satisfacerse la
  especificación? ¿Es insuficiente? ¿Redundante? ¿Contradictoria? ¿Qué
  restricciones se suponen sobre los argumentos y el resultado?
\item ¿Puedes descomponer el problema en partes? Puede ser útil dibujar
  diagramas con ejemplos de argumentos y resultados.
\end{itemize}

\subsubsection*{Paso 2: Diseñar el programa}
\begin{itemize}
\item ¿Te has encontrado con un problema semejante? ¿O has visto el mismo
  problema planteado en forma ligeramente diferente? 
\item ¿Conoces algún problema \emph{relacionado} con éste? ¿Conoces alguna
  función que te pueda ser útil? Mira atentamente el tipo y trata de recordar un
  problema que sea familiar y que tenga el mismo tipo o un tipo similar. 
\item ¿Conoces algún problema familiar con una \emph{especificación} similar? 
\item He aquí un problema \emph{relacionado} al tuyo y que ya has
  resuelto. ¿Puedes utilizarlo? ¿Puedes utilizar su resultado? ¿Puedes emplear
  su método? ¿Te hace falta introducir alguna función auxiliar a fin de poder
  utilizarlo?  
\item Si no puedes resolver el problema propuesto, trata de resolver primero
  algún problema similar. ¿Puedes imaginarte un problema análogo un
  tanto más \emph{accesible}? ¿Un problema más \emph{general}? ¿Un problema más
  \emph{particular}? ¿Un problema \emph{análogo}? 
\item ¿Puede resolver una \emph{parte} del problema? ¿Puedes deducir algún
  elemento útil de los datos? ¿Puedes pensar en algunos otros datos apropiados
  para determinar la incógnita? ¿Puedes cambiar la incógnita? ¿Puedes cambiar la
  incógnita o los datos, o ambos si es necesario, de tal forma que estén más
  cercanos entre sí? 
\item ¿Has empleado todos los datos? ¿Has empleado todas las restricciones
  sobre los datos? ¿Has considerado todas los requisitos de la especificación?
\end{itemize}

\subsubsection*{Paso 3: Escribir el programa}
\begin{itemize}
\item Al escribir el programa, comprueba cada uno de los pasos y funciones
  auxiliares. 
\item ¿Puedes ver claramente que cada paso o función auxiliar es correcta?
\item Puedes escribir el programa en \emph{etapas}. Piensas en los diferentes
  \emph{casos} en los que se divide el problema; en particular, piensas en los
  diferentes casos para los datos. Puedes pensar en el cálculo de los casos
  independientemente y \emph{unirlos} para obtener el resultado final
\item Puedes pensar en la solución del problema descomponiéndolo en problemas
  con datos más simples y uniendo las soluciones parciales para obtener la
  solución del problema; esto es, por \emph{recursión}.
\item En su diseño se puede usar problemas más generales o más
  particulares. Escribe las soluciones de estos problemas; ellas puede servir
  como guía para la solución del problema original, o se pueden usar en su
  solución. 
\item ¿Puedes apoyarte en otros problemas que has resuelto? ¿Pueden usarse?
  ¿Pueden modificarse? ¿Pueden guiar la solución del problema original?
\end{itemize}

\subsubsection*{Paso 4: Examinar la solución obtenida}
\begin{itemize}
\item ¿Puedes comprobar el funcionamiento del programa sobre una colección de
  argumentos? 
\item ¿Puedes comprobar propiedades del programa?
\item ¿Puedes escribir el programa en una forma diferente?
\item ¿Puedes emplear el programa o el método en algún otro programa? 
\end{itemize}

\noindent
Simon Thompson 
\href{http://www.cs.kent.ac.uk/people/staff/sjt/Haskell_craft/HowToProgIt.html}
     {\emph{How to program it}}, 
basado en G. Polya \emph{Cómo plantear y resolver problemas}.
